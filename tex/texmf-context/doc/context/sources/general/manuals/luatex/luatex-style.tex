\startenvironment luatex-style

% todo: use \useMPlibrary[lua]

\usemodule[abr-02]

\setuplayout
  [height=middle,
   width=middle,
   backspace=2cm,
   topspace=10mm,
   bottomspace=10mm,
   header=10mm,
   footer=10mm,
   footerdistance=10mm,
   headerdistance=10mm]

\setuppagenumbering
  [alternative=doublesided]

\setuptolerance
  [stretch,tolerant]

\setuptype
  [lines=hyphenated]

\setuptyping
  [lines=hyphenated]

\setupitemize
  [each]
  [packed]

\setupwhitespace
  [medium]

\startluacode
    local skipped = table.tohash { 'id', 'subtype', 'next', 'prev' }

    function document.functions.showfields(s)
        local t = string.split(s,',')
        local f = node.fields(t[1],t[2])
        if f then
            local d = false
            for i=1,#f do
                local fi = f[i]
                if skipped[fi] then
                    -- okay
                elseif d then
                    context(', {\tttf %s}', fi)
                else
                    context('{\tttf %s}', fi)
                    d = true
                end
            end
        end
    end

    function document.functions.showid(s)
        local t = string.split(s,',')
        context('{tttf %s}',node.id(t[1]))
        if t[2] then
            context(', {tttf %s}',node.subtype(t[2]))
        end
    end

    function document.functions.showsubtypes(s)
        local s = node.subtypes(s)
        local d = false
        for k, v in table.sortedhash(s) do
            if d then
                context(', %s = {\\tttf %s}',k,v)
            else
                context('%s = {\\tttf %s}',k,v)
                d = true
            end
        end
    end
\stopluacode

\unexpanded\def\showfields  #1{\ctxlua{document.functions.showfields("#1")}}
\unexpanded\def\showid      #1{\ctxlua{document.functions.showid("#1")}}
\unexpanded\def\showsubtypes#1{\ctxlua{document.functions.showsubtypes("#1")}}

\definecolor[blue]      [b=.5]
\definecolor[red]       [r=.5]
\definecolor[green]     [g=.5]
\definecolor[maincolor] [b=.5]
\definecolor[keptcolor] [b=.5]
\definecolor[othercolor][r=.5,g=.5]

\setupbodyfont[modern] % we need this in examples so we predefine

% \doifmodeelse {atpragma} {
%
%   %  \setupbodyfont
%   %    [lucidaot,10pt]
%
%     \setupbodyfont
%       [dejavu,10pt]
%
%     \setuphead [chapter]      [style=\bfd]
%     \setuphead [section]      [style=\bfb]
%     \setuphead [subsection]   [style=\bfa]
%     \setuphead [subsubsection][style=\bf]
%
% } {
%
%     \definetypeface[mainfacenormal]  [ss][sans] [iwona]       [default]
%     \definetypeface[mainfacenormal]  [rm][serif][palatino]    [default]
%     \definetypeface[mainfacenormal]  [tt][mono] [modern]      [default][rscale=1.1]
%     \definetypeface[mainfacenormal]  [mm][math] [iwona]       [default]
%
%     \definetypeface[mainfacemedium]  [ss][sans] [iwona-medium][default]
%     \definetypeface[mainfacemedium]  [rm][serif][palatino]    [default]
%     \definetypeface[mainfacemedium]  [tt][mono] [modern]      [default][rscale=1.1]
%     \definetypeface[mainfacemedium]  [mm][math] [iwona-medium][default]
%
%     \setupbodyfont
%       [mainfacenormal,10pt]
%
%     \setuphead [chapter]      [style=\mainfacemedium\bfd]
%     \setuphead [section]      [style=\mainfacemedium\bfb]
%     \setuphead [subsection]   [style=\mainfacemedium\bfa]
%     \setuphead [subsubsection][style=\mainfacemedium\bf]
%
% }

\writestatus{luatex manual}{we assume that dejavu math is available}

\setupbodyfont % assumes dejavu-math
  [dejavu,10pt]

\setuphead [chapter]      [align={flushleft,broad},style=\bfd]
\setuphead [section]      [align={flushleft,broad},style=\bfb]
\setuphead [subsection]   [align={flushleft,broad},style=\bfa]
\setuphead [subsubsection][align={flushleft,broad},style=\bf]

\setuphead [chapter]      [color=maincolor]
\setuphead [section]      [color=maincolor]
\setuphead [subsection]   [color=maincolor]
\setuphead [subsubsection][color=maincolor]

\definehead
  [remark]
  [subsubsubject]

\setupheadertexts
  []

\definemixedcolumns
  [twocolumns]
  [n=2,
   balance=yes,
   before=\blank,
   after=\blank]

\definemixedcolumns
  [threecolumns]
  [twocolumns]
  [n=3]

\definemixedcolumns
  [fourcolumns]
  [threecolumns]
  [n=4]

% if we do this we also need to do it in table cells
%
% \setuptyping
%   [color=maincolor]
%
% \setuptype
%   [color=maincolor]

\definetyping
  [functioncall]

\startMPdefinitions

    color   luaplanetcolor ; luaplanetcolor := \MPcolor{maincolor} ;
    color   luaholecolor   ; luaholecolor   := white ;
    numeric luaextraangle  ; luaextraangle  := 0 ;
    numeric luaorbitfactor ; luaorbitfactor := .25 ;

    vardef lualogo = image (

        % Graphic design by A. Nakonechnyj. Copyright (c) 1998, All rights reserved.

        save d, r, p ; numeric d, r, p ;

        d := sqrt(2)/4 ; r := 1/4 ; p := r/8 ;

        fill fullcircle scaled 1
            withcolor luaplanetcolor ;
        draw fullcircle rotated 40.5 scaled (1+r)
            dashed evenly scaled p
            withpen pencircle scaled (p/2)
            withcolor (luaorbitfactor * luaholecolor) ;
        fill fullcircle scaled r shifted (d+1/8,d+1/8)
            rotated - luaextraangle
            withcolor luaplanetcolor ;
        fill fullcircle scaled r shifted (d-1/8,d-1/8)
            withcolor luaholecolor   ;
        luaorbitfactor := .25 ;
    )  enddef ;

\stopMPdefinitions

\startuseMPgraphic{luapage}
    StartPage ;

        fill Page withcolor \MPcolor{othercolor} ;

        luaorbitfactor := 1 ;
        picture p ; p := lualogo xsized (3PaperWidth/5) ;
        draw p shifted center Page shifted (0,-.5ypart center ulcorner p) ;

    StopPage ;
\stopuseMPgraphic

% \starttexdefinition luaextraangle
%     % we can also just access the last page and so in mp directly
%     \ctxlua {
%         context(\lastpage == 0 and 0 or \realfolio*360/\lastpage)
%     }
% \stoptexdefinition

\startuseMPgraphic{luanumber}
  % luaextraangle  := \luaextraangle;
    luaextraangle  := if (LastPageNumber == 0) : 0 else : (RealPageNumber / LastPageNumber) * 360  fi;
    luaorbitfactor := 0.25 ;
    picture p ; p := lualogo ;
    setbounds p to boundingbox fullcircle ;
    draw p ysized 1cm ;
\stopuseMPgraphic

\definelayer
  [page]
  [width=\paperwidth,
   height=\paperheight]

\setupbackgrounds
  [leftpage]
  [background=page]

\setupbackgrounds
  [rightpage]
  [background=page]

\startsetups pagenumber:right
  \setlayerframed
    [page]
    [preset=rightbottom,offset=1cm]
    [frame=off,height=1cm,offset=overlay]
    {\useMPgraphic{luanumber}}
  \setlayerframed
    [page]
    [preset=rightbottom,offset=1cm,x=1.5cm]
    [frame=off,height=1cm,width=1cm,offset=overlay]
    {\pagenumber}
  \setlayerframed
    [page]
    [preset=rightbottom,offset=1cm,x=2.5cm]
    [frame=off,height=1cm,offset=overlay]
    {\getmarking[chapter]}
\stopsetups

\startsetups pagenumber:left
  \setlayerframed
    [page]
    [preset=leftbottom,offset=1cm,x=2.5cm]
    [frame=off,height=1cm,offset=overlay]
    {\getmarking[chapter]}
  \setlayerframed
    [page]
    [preset=leftbottom,offset=1cm,x=1.5cm]
    [frame=off,height=1cm,width=1cm,offset=overlay]
    {\pagenumber}
  \setlayerframed
    [page]
    [preset=leftbottom,offset=1cm]
    [frame=off,height=1cm,offset=overlay]
    {\useMPgraphic{luanumber}}
\stopsetups

\unexpanded\def\nonterminal#1>{\mathematics{\langle\hbox{\rm #1}\rangle}}

% taco's brainwave -)

\newcatcodetable\syntaxcodetable

\unexpanded\def\makesyntaxcodetable
  {\begingroup
   \catcode`\<=13 \catcode`\|=12
   \catcode`\!= 0 \catcode`\\=12
   \savecatcodetable\syntaxcodetable
   \endgroup}

\makesyntaxcodetable

\unexpanded\def\startsyntax {\begingroup\catcodetable\syntaxcodetable  \dostartsyntax}
\unexpanded\def\syntax      {\begingroup\catcodetable\syntaxcodetable  \dosyntax}
           \let\stopsyntax   \relax

\unexpanded\def\syntaxenvbody#1%
  {\par
   \tt
   \startnarrower
   \maincolor #1
   \stopnarrower
   \par}

\unexpanded\def\syntaxbody#1%
  {\begingroup
   \maincolor \tt #1%
   \endgroup}

\bgroup \catcodetable\syntaxcodetable

!gdef!dostartsyntax#1\stopsyntax{!let<!nonterminal!syntaxenvbody{#1}!endgroup}
!gdef!dosyntax     #1{!let<!nonterminal!syntaxbody{#1}!endgroup}

!egroup

% end of wave

\setupinteraction
  [state=start,
   focus=standard,
   style=,
   color=,
   contrastcolor=]

\placebookmarks
  [chapter,section,subsection]

\setuplist
  [chapter,section,subsection,subsubsection]
  [interaction=all,
   width=3em]

\setuplist
  [chapter]
  [style=bold,
   color=keptcolor]

\setuplist
  [subsection,subsubsection]
  [margin=3em,
   width=5em]

% Hans doesn't like the bookmarks opening by default so we comment this:
%
% \setupinteractionscreen
%   [option=bookmark]

\stopenvironment
