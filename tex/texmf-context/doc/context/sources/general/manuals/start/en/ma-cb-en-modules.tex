\startcomponent ma-cb-en-modules

\enablemode[**en-us]

\project ma-cb

\startchapter[title=Using modules]

%% VZ: many new modules was added (see ma-cb-cz-modules.tex)

\index{module}

\Command{\tex{usemodule}}

For reasons of efficiency \CONTEXT\ comes with a number of modules that contain
specific functionality. Loading a module is done in the set up area of your input
file by means of:

\shortsetup{usemodule}

When you load a module \CONTEXT\ looks for a file with the following (prefix-)name:

\startitemize[packed]
\item m-modulename (core module)
\item p-modulename (private module)
\item s-modulename (\CONTEXT\ style file)
\item x-modulename (XML module)
\item t-modulename (third party module)
\item modulename
\stopitemize

A few example core modules are:

\startitemize[packed]
\item m-fields        (\type{m-fields.mkiv}): for PDF forms
\item m-morse         (\type{m-morse.mkvi}): for morse
\item m-spreadsheet   (\type{m-spreadsheet.mkiv}): for spreadsheets
\item m-visual        (\type{m-visual.mkiv}): for visual debugging
\item m-zint          (\type{m-zint.mkiv}): for generating bar codes
\item s-pre-**        (\type{s-pre-**}): for presentations
\stopitemize

\stopchapter

\stopcomponent
