% This file is a solution template for:

% - Giving a talk on some subject.
% - The talk is between 15min and 45min long.

% This is adapted from the example by Till Tantau <tantau@users.sourceforge.net>
% included as part of the beamer package in LaTeX
%
% In principle, this file can be redistributed and/or modified under
% the terms of the GNU Public License, version 2.
%
% However, this file is supposed to be a template to be modified
% for your own needs. For this reason, if you use this file as a
% template and not specifically distribute it as part of a another
% package/program, the author grants the extra permission to freely
% copy and  modify this file as you see fit and even to delete this
% copyright notice. 

\usemodule
  [simpleslides]
  [%%%%%%%%%%%%%%%%%%%%%
   % Choose a style
  %%%%%%%%%%%%%%%%%%%%%%
    style=BigNumber,            % color=blue|red
  % style=BottomSquares,
  % style=Boxed,
  % style=Ellipse,
  % style=Embossed,
  % style=Framed,               % alternative=square|stripe
  % style=FramedTitle,
  % style=HorizontalStripes,    % color=blue|green|red
  % style=NarrowStripes,        % color=blue|green|red
  % style=RainbowStripe,
  % style=Rounded,
  % style=Shaded,               % color=blue|green|bluered
  % style=SideSquares,
  % style=Split,
  % style=Sunrise,
  % style=Swoosh,
  % style=ThickStripes,
  %%%%%%%%%%%%%%%%%%%%%%%
   % Choose a font
  %%%%%%%%%%%%%%%%%%%%%%%
    font=Bookman,
  % font=Chancery,
  % font=Gothic,
  % font=Helevetica,
  % font=LatinModern,
  % font=LatinModernSans,
  % font=Palatino,
  % font=Schoolbook,
  % font=Times,
  %%%%%%%%%%%%%%%%%%%%%%%%%
    % Choose a font size
  %%%%%%%%%%%%%%%%%%%%%%%%%
    size=17pt, % Default is 17pt
  %%%%%%%%%%%%%%%%%%%%%%%%%
    % Choose a color.
    % Only a few styles support color
  %%%%%%%%%%%%%%%%%%%%%%%%%
    color=blue, % supported by:
    % BigNumber, HorizontalStripes, NarrowStripes, Shaded
  % color=red, % supported by:
    % BigNumber, HorizontalStripes, NarrowStripes
  % color=green, % supported by:
    % HorizontalStripes, NarrowStripes, Shaded
  %%%%%%%%%%%%%%%%%%%%%%%%%
    % Choose a alternative.
    % Only one style support alternative
  %%%%%%%%%%%%%%%%%%%%%%%%%
    alternative=square, % supported by:
    % Framed
  % alternative=stripe, % supported by:
    % Framed
  ]



\setupTitle
  [title={Presentation Title},
   author={F.~Author, S.~Another},
   date={Date / Occasion}]

% In order to use sample images distributed with ConTeXt
\setupexternalfigures[location={local,global,default}] 

\starttext

\placeTitle

% Since this a solution template for a generic talk, very little can
% be said about how it should be structured. However, the talk length
% of between 15min and 45min and the theme suggest that you stick to
% the following rules:  

% - Exactly two or three sections (other than the summary).
% - At *most* three subsections per section.
% - Talk about 30s to 2min per slide. So there should be between about
%   15 and 30 slides, all told.


\SlideTitle{Make Titles Informative}
% - A title should summarize the slide in an understandable fashion
%   for anyone how does not follow everything on the slide itself.

\startitemize
  \item Use bullets points when appropriate.
  \item Use pictures when possible
  \item Do not put too much information on one slide
\stopitemize

% You can include full slide pictures. There are some helper commands to
% decorate the picture. 

% To include a full slide picute.
\IncludePicture
  [horizontal]
  [cow] % Name of the image
  {A Dutch Cow} % Title of the slide

% We can decorate the picture using circles, arrows, and focus. In order to
% determine where to draw the decorations, it is helpful to draw a grid.

\IncludePicture
  [horizontal]
  [cow] % Name of the image
  [highlight=yes,
   grid=yes]
  {A Dutch Cow with a grid} % Title of the slide

% Sometimes you need a finer grip to fine tune the exact position
\IncludePicture
  [horizontal]
  [cow] % Name of the image
  [highlight=yes,
   grid=yes,
   steps=5, % Each grid block is broken into these many parts.
   subgrid=yes]
  {A Dutch Cow with a fine grid} % Title of the slide

% Now lets try to draw a circle around the cow's head
\IncludePicture
  [horizontal]
  [cow] % Name of the image
  [highlight=yes,
   grid=yes,
   subgrid=yes,
   alternative=circle,
   color=orange,
   x=1.4,
   y=8.2,
   xscale=1.5,
   yscale=couple, % To ensure that we get a circle and not an ellipse
   shadow=bottomleft]
  {The head of a dutch cow} % Title of the slide


% Now that we have found the correct location to place the circle, we can remove
% the helper grid.
\IncludePicture
  [horizontal]
  [cow] % Name of the image
  [highlight=yes,
   grid=no,
   subgrid=no,
   alternative=circle,
   color=orange,
   x=1.4,
   y=8.2,
   xscale=1.5,
   yscale=couple, % To ensure that we get a circle
   shadow=bottomleft]
   {The head of a dutch cow} % Title of the slide

% We can also draw arrows.
\IncludePicture
  [horizontal]
  [cow] % Name of the image
  [highlight=yes,
   grid=no,
   subgrid=no,
   alternative=arrow,
   color=orange,
   x=0.4,
   y=6.8,
   direction=-90, 
   length=3cm,
   shadow=topright] % The shadow direction gets rotated with the arrow
   {The mouth of a dutch cow} % Title of the slide

% We can also highlight specific parts of the picture
\IncludePicture
  [horizontal]
  [cow] % Name of the image
  [highlight=yes,
   grid=no,
   subgrid=no,
   alternative=focus,
   color=orange,
   x=1.4,
   y=8.2,
   xscale=1.5,
   yscale=couple, % To ensure that we get a circle
   opacity=0.5]
   {The head of a dutch cow} % Title of the slide

% You can also include a picture with some text along side it.

\IncludePicture
  [vertical]
  [mill] % Name of the image
  {The windmills are an example of a green energy source.} % Slide title

\SlideTitle{Summary}

\startitemize
  \item The {\em first main message} of your talk in one or two lines.
  \item The {\em second main message} of your talk in one or two lines.
  \item Perhaps a {\em third message}, but not more than that.
\stopitemize

\stoptext

