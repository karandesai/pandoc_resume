%D \module
%D   [      file=simpleslides-s-Ellipse,
%D        version=2009.03.30
%D          title=\CONTEXT\ Style File,
%D       subtitle=Presentation Module Ellipse,
%D         author=Aditya Mahajan and Thomas A. Schmitz,
%D           date=\currentdate,
%D      copyright={Aditya Mahajan and Thomas A. Schmitz}]
%C
%C Copyright 2007 Aditya Mahajan and Thomas A. Schmitz
%C This file may be distributed under the GNU General Public License v. 2.0.

%D This file provides the \quotation{Ellipse} style for the presentation
%D module. The design is inspired by Hans's "funny" presentation module
%D (s-pre-03). It is loaded at runtime. 

\writestatus{simpleslides}{loading Ellipse style}

\startmodule[simpleslides-s-Ellipse]

\unprotect

%D First, we change the page layout.

\setuplayout [width=fit,
              height=fit,
	      margin=0cm,
              header=1.2cm, 
              footer=0cm, 
              topspace=1.8cm, 
              backspace=1.5cm,
              location=singlesided]

\setuplayout [simpleslides:layout:horizontal][header=15mm]
\setuplayout [simpleslides:layout:vertical]  [header=0mm]
\setuplayout [simpleslides:layout:title]     [header=0mm]

%D These macros are used for placing figures/pictures:

\define\NormalHeight            {\textheight}
\define\NormalWidth             {.5\textwidth}
\define\PictureFrameHeight      {\textheight}
\define\PictureFrameWidth       {.5\textwidth}

\setuplayer
  [simpleslides:layer:slidetitle]
    [y=8mm,
     x=15mm]

%D We define our color scheme:

\definecolor[simpleslides:variantcolor]              [s=0]
\definecolor[simpleslides:backgroundcolor]           [s=.8]
\definecolor[simpleslides:contrastcolor]             [r=.5,g=0,b=0]
\definecolor[simpleslides:altcontrastcolor]          [r=.9,g=0,b=0]
\definecolor[simpleslides:itemize:color]             [r=.5]

%D We let Metapost calculate the background:

\startuseMPgraphic{simpleslides:MP:horizontal}
StartPage ;
  fill Page withcolor \MPcolor{simpleslides:variantcolor} ;
  save p ; path p ; 
  p := Page enlarged (-15pt,-15pt) superellipsed .9 ;
  fill p withcolor \MPcolor{simpleslides:backgroundcolor} ;
  pickup pencircle scaled 20pt ;
  draw p withcolor \MPcolor{simpleslides:contrastcolor} ;
StopPage ;
\stopuseMPgraphic

\startuseMPgraphic{simpleslides:MP:ornament} 
StartPage ;
 save p ; path p ;
 p := Page enlarged (-15pt,-15pt) superellipsed .9 ;
 pickup pencircle scaled 20pt ;
 save pa, pb; pair pa, pb ;
 if PageNumber>1:
  pa := point (3 + (6*PageNumber) / NOfPages) of p ;
  pb := point (3 + (6*(PageNumber-1)) / NOfPages) of p ; 
  draw (p cutafter pa) cutbefore pb 
      withcolor \MPcolor{simpleslides:altcontrastcolor} ;
 fi ;
StopPage ;
\stopuseMPgraphic 

%D We define these backgrounds as overlays:

\defineoverlay 
  [simpleslides:background:horizontal] 
  [\useMPgraphic{simpleslides:MP:horizontal}] 

\defineoverlay 
  [simpleslides:background:vertical] 
  [\useMPgraphic{simpleslides:MP:horizontal}] 

\defineoverlay 
  [simpleslides:background:title] 
  [\useMPgraphic{simpleslides:MP:horizontal}] 

\defineoverlay 
  [simpleslides:background:ornament] 
  [\useMPgraphic{simpleslides:MP:ornament}] 

%D We want the title to placed in color. 

\setupTitle[\c!headcolor={simpleslides:contrastcolor}]

%D We want the slide title on the top

\setupSlideTitle
   [\c!after=,
    \c!alternative=layer,
    \c!width=\textwidth,
    \c!align=\v!center,
    \c!height=3.5cm,
    \c!color=simpleslides:contrastcolor]

\setupcombinations[distance=0cm]

%D The symbol for the first level of itemizations. 

\definesymbol[1][\useMPgraphic{simpleslides:itemize:square}]
\setupitemize[1][\c!color={simpleslides:itemize:color}]

\protect
\stopmodule

\endinput

