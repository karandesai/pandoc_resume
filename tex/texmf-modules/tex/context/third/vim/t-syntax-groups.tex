%D \module
%D   [     file=t-syntax-groups,
%D      version=2012.05.06,
%D        title=\CONTEXT\ User Module,
%D     subtitle=Syntax highlighting support,
%D       author=Aditya Mahajan,
%D         date=\currentdate,
%D    copyright=Aditya Mahajan,
%D        email=adityam <at> ieee <dot> org,
%D      license=Simplified BSD License]

\writestatus{loading}{Syntax highlighting groups (ver: 2012.05.06)}

\startmodule [syntax-group]
\usemodule   [module-catcodes]

\unprotectmodulecatcodes

% Colors are specified in hex; in MkII the hex mode needs to be activated.
\doifmode\s!mkii
    {\setupcolor[hex]}


\def\colorscheme@name {}

\def\syntaxgroup@id          {syntaxgroup}
\def\syntaxgroup@namespace   {@@@@\syntaxgroup@id}
\def\syntaxgroup@name        {}

\edef\t!syntaxgroup           {\syntaxgroup@id}


\installparameterhandler  \syntaxgroup@namespace \syntaxgroup@id
\installsetuphandler      \syntaxgroup@namespace \syntaxgroup@id
\doifmode{\s!mkiv}
  {\installstyleandcolorhandler      \syntaxgroup@namespace \syntaxgroup@id}

\def\m!syntaxgroup{t-syntax-group}

\def\definesyntaxgroup
    {\dodoubleargument\syntaxgroup@define}

\starttexdefinition syntaxgroup@define [#1][#2]
  % #1 list name
  % #2 options
  \doifassignmentelse{#2}
  {
    \def\syntaxgroup@get_parameters##1%
    {
         \edef\syntaxgroup@name {\colorscheme@name##1}
         \setupsyntaxgroup[\syntaxgroup@name]
                          [\s!parent=\syntaxgroup@namespace,\c!color=,\c!style=,\c!command=,#2]
         \doifsomething{\syntaxgroupparameter\c!color}
          {
             \expanded{\definecolor[\syntaxgroup@namespace-##1-color]
                                   [\syntaxgroupparameter\c!color]}
             \setupsyntaxgroup[\syntaxgroup@name][\s!parent=\syntaxgroup@namespace,\c!color=\syntaxgroup@namespace-##1-color]
          }
    }
  }{
    \def\syntaxgroup@get_parameters##1%
    {
        \edef\syntaxgroup@name {\colorscheme@name##1}
        \expanded{\setupsyntaxgroup[\syntaxgroup@name]
                         [\c!color=\namedsyntaxgroupparameter{\colorscheme@name#2}\c!color,
                          \c!style=\namedsyntaxgroupparameter{\colorscheme@name#2}\c!style,
                        \c!command=\namedsyntaxgroupparameter{\colorscheme@name#2}\c!command,
                         ]}

        % In MkII, \expanded messes up the definition of \currentsyntaxgroup
        \def\currentsyntaxgroup   {\syntaxgroup@name}
    }
  }

  \processcommalist[#1]\syntaxgroup@get_parameters
\stoptexdefinition

\def\startcolorscheme%
    {\dosingleargument\colorscheme@start}

\starttexdefinition colorscheme@start [#1]
     \pushmacro\colorscheme@name
     \setcolorscheme{#1}
     %\setupsyntaxgroup[\c!color=,\c!style=,\c!command=]
\stoptexdefinition

\def\stopcolorscheme
    {\popmacro\colorscheme@name}

\def\setcolorscheme#1%
    {\edef\colorscheme@name{#1}}

\starttexdefinition syntaxgroup [#1]#2
  % #1 = style
  % #2 = content
      \begingroup
      \edef\syntaxgroup@name{\colorscheme@name#1}%
      \syntaxgroupparameter\c!before
      \iftracesyntaxgroups
          \syntaxgroup@show_values
      \fi
      \doifmode{\s!mkiv}{\dostarttagged\t!syntaxgroup{#1}}
      \syntaxgroupparameter\c!command
      {
        \externalfilter@attributes_start\syntaxgroup@id\c!style\c!color
            #2
        \externalfilter@attributes_stop
      }
      \doifmode{\s!mkiv}{\dostoptagged}
      \syntaxgroupparameter\c!after
      \endgroup
\stoptexdefinition

%D Tracing macros

\newif\iftracesyntaxgroups
\let\tracesyntaxgroups\tracesyntaxgroupstrue

\def\currentsyntaxgroup   {\syntaxgroup@name}

\starttexdefinition syntaxgroup@show_values
  \writestatus\m!syntaxgroup{color scheme : \colorscheme@name}
  \writestatus\m!syntaxgroup{current group: \syntaxgroup@name}
  \writestatus\m!syntaxgroup{command      : \syntaxgroupparameter\c!command}
  \writestatus\m!syntaxgroup{style        : \syntaxgroupparameter\c!style}
  \writestatus\m!syntaxgroup{color        : \syntaxgroupparameter\c!color}
\stoptexdefinition

\protectmodulecatcodes
\stopmodule
